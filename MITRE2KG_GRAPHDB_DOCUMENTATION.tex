\documentclass[12pt,a4paper,openany]{report}

% Packages
\usepackage[utf-8]{inputenc}
\usepackage[english]{babel}
\usepackage[margin=1in]{geometry}
\usepackage{hyperref}
\usepackage{graphicx}
\usepackage{tabularx}
\usepackage{booktabs}
\usepackage{listings}
\usepackage{xcolor}
\usepackage{float}
\usepackage{fancyhdr}
\usepackage{amsmath}
\usepackage{amssymb}
\usepackage{tikz}
\usepackage{array}
\usepackage{multirow}
\usepackage{longtable}
\usepackage{parskip}

% Code listing style
\lstset{
    basicstyle=\ttfamily\small,
    breaklines=true,
    commentstyle=\color{gray},
    keywordstyle=\color{blue},
    numberstyle=\tiny\color{gray},
    stringstyle=\color{orange},
    showstringspaces=false,
    tabsize=2,
    language=python,
    backgroundcolor=\color{gray!10},
    frame=lines,
    captionpos=b,
    lineskip=-0.5em
}

% Header and Footer
\pagestyle{fancy}
\fancyhf{}
\fancyhead[L]{\textit{MITRE2KG Graph Database}}
\fancyhead[R]{\thepage}
\fancyfoot[C]{\footnotesize Generated: December 23, 2025}
\renewcommand{\headrulewidth}{0.4pt}

% Title Page
\title{\Huge \textbf{MITRE2KG Graph Database} \\
        \Large Schema, Architecture \& Implementation \\
        \vspace{0.5cm}
        \normalsize A Unified Knowledge Graph of MITRE ATT\&CK Framework \\
        and CISA Cybersecurity Advisories}

\author{Cybersecurity Intelligence System \\ 
        ArangoDB Integration with Semantic Embeddings}

\date{December 23, 2025}

\begin{document}

% Title Page
\maketitle

% Abstract
\begin{abstract}
This document provides comprehensive documentation of the MITRE2KG graph database system, 
which integrates the MITRE ATT\&CK Enterprise Framework with real-world CISA cybersecurity 
advisories. The system leverages ArangoDB as a multi-model graph database, semantic embeddings 
for similarity search, and Ollama LLM for intelligent threat analysis. This document covers 
the complete architecture, schema design, data generation process, triplet extraction methodology, 
and provides detailed statistics of the final integrated knowledge graph.

\begin{itemize}
    \item \textbf{Total Entities:} 24,556
    \item \textbf{Total Relationships:} 24,342
    \item \textbf{Attack Patterns:} 823 (345 parent + 478 subtechniques)
    \item \textbf{CISA Advisories:} 74 with 1,605 technique links
    \item \textbf{Semantic Embeddings:} 47,293 vectors (384-dimensional)
    \item \textbf{Integration Success Rate:} 94.4\%
\end{itemize}
\end{abstract}

\tableofcontents
\newpage

% ============================================================================
\chapter{Executive Summary}
% ============================================================================

The MITRE2KG project represents a significant advancement in cybersecurity threat intelligence 
by creating a unified, machine-readable knowledge graph that combines two critical data sources:

\section{System Architecture Diagram}

\begin{figure}[H]
\centering
\begin{tikzpicture}[node distance=2cm, auto, scale=0.9]
    % Data Sources Layer
    \node[draw, rectangle, fill=blue!20, minimum width=3cm, minimum height=1cm] (mitre) {MITRE\\enterprise-attack.json};
    \node[draw, rectangle, fill=blue!20, minimum width=3cm, minimum height=1cm, right=of mitre] (cisa) {CISA\\CISA-crawl-rt-ttp-ct.csv};
    
    % Processing Layer
    \node[draw, rectangle, fill=green!20, minimum width=3cm, minimum height=0.8cm, below=of mitre, yshift=-1.5cm] (extract) {Data Extraction\\& Parsing};
    \node[draw, rectangle, fill=green!20, minimum width=3cm, minimum height=0.8cm, right=of extract] (normalize) {Entity\\Normalization};
    \node[draw, rectangle, fill=green!20, minimum width=3cm, minimum height=0.8cm, right=of normalize] (triplet) {Triplet\\Generation};
    
    % Storage Layer
    \node[draw, rectangle, fill=orange!20, minimum width=2.2cm, minimum height=1.2cm, below=of extract, yshift=-2cm] (entities) {entities\\(24,556)};
    \node[draw, rectangle, fill=orange!20, minimum width=2.2cm, minimum height=1.2cm, right=of entities] (rels) {relationships\\(24,342)};
    \node[draw, rectangle, fill=orange!20, minimum width=2.2cm, minimum height=1.2cm, right=of rels] (embed) {embeddings\\(47,293)};
    
    % Database
    \node[draw, rectangle, fill=orange!30, minimum width=9cm, minimum height=1cm, below=of rels, yshift=-2cm] (db) {ArangoDB: MITRE2kg};
    
    % Application Layer
    \node[draw, rectangle, fill=red!20, minimum width=3.5cm, minimum height=1cm, below=of db, yshift=-2cm] (rag) {RAG Query Engine};
    \node[draw, rectangle, fill=red!20, minimum width=3.5cm, minimum height=1cm, right=of rag] (llm) {Ollama LLM\\(llama3.1:8b)};
    
    % Arrows
    \draw[->] (mitre) -- (extract);
    \draw[->] (cisa) -- (extract);
    \draw[->] (extract) -- (normalize);
    \draw[->] (normalize) -- (triplet);
    \draw[->] (triplet) -- (entities);
    \draw[->] (triplet) -- (rels);
    \draw[->] (triplet) -- (embed);
    \draw[->] (entities) -- (db);
    \draw[->] (rels) -- (db);
    \draw[->] (embed) -- (db);
    \draw[->] (db) -- (rag);
    \draw[->] (db) -- (llm);
\end{tikzpicture}
\caption{MITRE2KG System Architecture}
\end{figure}

\begin{enumerate}
    \item \textbf{MITRE ATT\&CK Enterprise Framework} --- A comprehensive, standardized knowledge 
    base containing 823 attack techniques, organized into 14 tactics, with relationships 
    showing which threat groups employ which techniques.
    
    \item \textbf{CISA Cybersecurity Advisories} --- 77 real-world cybersecurity advisories 
    covering 74 unique incident response cases, with 432 unique attack techniques referenced 
    across 1,701 total technique mentions.
\end{enumerate}

By integrating these sources through a knowledge graph, the system enables:

\begin{description}
    \item[Semantic Search] Finding attack techniques by meaning, not keywords
    \item[Threat Intelligence] Connecting real-world attacks to MITRE framework techniques
    \item[Graph Traversal] Discovering related techniques, threat actors, and attack chains
    \item[AI-Powered Analysis] Leveraging Ollama LLM with graph context for intelligent responses
    \item[Incident Response] Quick lookup of techniques and advisories related to incidents
\end{description}

\section{System Architecture Overview}

The MITRE2KG system consists of four integrated layers:

\begin{itemize}
    \item \textbf{Data Layer:} MITRE enterprise-attack.json (14,000+ objects) and CISA-crawl-rt-ttp-ct.csv (77 rows)
    \item \textbf{Storage Layer:} ArangoDB with 4 main collections (entities, relationships, entity\_embeddings, relationship\_embeddings)
    \item \textbf{Processing Layer:} Python-based ETL pipelines for data extraction, transformation, and embedding generation
    \item \textbf{Application Layer:} RAG Query Engine + Ollama LLM for semantic search and intelligent responses
\end{itemize}

\newpage

% ============================================================================
\chapter{Database Schema}
% ============================================================================

\section{Entity-Relationship Overview}

\begin{figure}[H]
\centering
\begin{tikzpicture}[node distance=2.5cm, auto, scale=0.8]
    % Entity types
    \node[draw, rectangle, fill=blue!20, minimum width=2cm, minimum height=0.7cm] (tactic) {Tactic};
    \node[draw, rectangle, fill=blue!20, minimum width=2cm, minimum height=0.7cm, below left=of tactic] (technique) {Technique};
    \node[draw, rectangle, fill=blue!20, minimum width=2cm, minimum height=0.7cm, below right=of tactic] (subtechnique) {Sub-Tech};
    \node[draw, rectangle, fill=green!20, minimum width=2cm, minimum height=0.7cm, below=of technique] (group) {Threat Group};
    \node[draw, rectangle, fill=green!20, minimum width=2cm, minimum height=0.7cm, right=of group] (malware) {Malware};
    \node[draw, rectangle, fill=red!20, minimum width=2cm, minimum height=0.7cm, below=of group] (advisory) {CISA Advisory};
    
    % Relationships
    \draw[->] (technique) -- node[above, font=\small] {belongs-to} (tactic);
    \draw[->] (subtechnique) -- node[above, font=\small] {subtechnique-of} (technique);
    \draw[->] (group) -- node[left, font=\small] {uses} (technique);
    \draw[->] (malware) -- node[right, font=\small] {uses} (technique);
    \draw[->] (technique) -- node[below, font=\small] {uses\_in\_cisa} (advisory);
    
\end{tikzpicture}
\caption{Simplified Entity-Relationship Diagram}
\label{fig:erd}
\end{figure}

\section{Overview}

MITRE2KG uses ArangoDB, a multi-model database system that combines document and graph database 
capabilities. The database is named \texttt{MITRE2kg} and is located at \texttt{http://localhost:8529} 
with credentials \texttt{root/openSesame}.

\section{Collections}

\subsection{entities Collection}

The primary document collection storing all entities extracted from MITRE and CISA sources.

\begin{longtable}{|p{4cm}|p{3cm}|p{7cm}|}
    \caption{Entity Types and Counts} \\
    \toprule
    \textbf{Entity Type} & \textbf{Count} & \textbf{Description} \\
    \midrule
    \endfirsthead
    \multicolumn{3}{c}{\textit{continued from previous page}} \\
    \toprule
    \textbf{Entity Type} & \textbf{Count} & \textbf{Description} \\
    \midrule
    \endhead
    \midrule
    \multicolumn{3}{r}{\textit{continued on next page}} \\
    \endfoot
    \endlastfoot
    
    relationship & 20,048 & Internal relationship objects \\
    x-mitre-analytic & 1,739 & MITRE analytic definitions \\
    attack-pattern & 823 & Core attack techniques \\
    malware & 695 & Malware families and variants \\
    x-mitre-detection & 691 & Detection strategies \\
    intrusion-set & 178 & Threat actor groups \\
    x-mitre-data-component & 106 & Data components for detection \\
    tool & 91 & Adversary tools and utilities \\
    cisa\_advisory & 74 & CISA cybersecurity advisories \\
    campaign & 52 & Organized campaigns \\
    course-of-action & 44 & Mitigation measures \\
    x-mitre-tactic & 14 & Attack tactics/phases \\
    x-mitre-matrix & 1 & Framework matrix definition \\
    \midrule
    \textbf{TOTAL} & \textbf{24,556} & \\
\end{longtable}

\subsection{relationships Collection}

Directed edges connecting entities. Each relationship represents a semantic connection between 
two entities with a specified relationship type.

\begin{longtable}{|p{4cm}|p{3cm}|p{7cm}|}
    \caption{Relationship Types and Counts} \\
    \toprule
    \textbf{Relationship Type} & \textbf{Count} & \textbf{Description} \\
    \midrule
    \endfirsthead
    \multicolumn{3}{c}{\textit{continued from previous page}} \\
    \toprule
    \textbf{Relationship Type} & \textbf{Count} & \textbf{Description} \\
    \midrule
    \endhead
    \midrule
    \multicolumn{3}{r}{\textit{continued on next page}} \\
    \endfoot
    \endlastfoot
    
    uses & 17,270 & Threat actors use techniques \\
    referenced\_in\_cisa\_advisory & 1,619 & Advisory references technique \\
    uses\_in\_cisa\_advisory & 1,605 & Technique in advisory attack \\
    mitigates & 1,445 & Countermeasure mitigates attack \\
    belongs-to-tactic & 1,071 & Technique belongs to tactic \\
    detects & 691 & Detection catches technique \\
    subtechnique-of & 476 & Sub-technique of parent \\
    revoked-by & 140 & Technique revoked and replaced \\
    attributed-to & 25 & Attribution relationship \\
    \midrule
    \textbf{TOTAL} & \textbf{24,342} & \\
\end{longtable}

\subsection{entity\_embeddings Collection}

Dense semantic vectors (embeddings) for all entities. Enables similarity-based search and 
semantic understanding of entities.

\begin{table}[H]
    \centering
    \caption{entity\_embeddings Collection Properties}
    \begin{tabular}{|l|l|}
        \hline
        \textbf{Property} & \textbf{Value} \\
        \hline
        Total Documents & 24,556 \\
        \hline
        Embedding Model & all-MiniLM-L6-v2 \\
        \hline
        Vector Dimensions & 384 \\
        \hline
        Normalization & L2 (unit norm) \\
        \hline
        Memory per Vector & 1.5 KB (float32) \\
        \hline
        Total Memory & $\sim$ 37 MB \\
        \hline
    \end{tabular}
\end{table}

\subsection{relationship\_embeddings Collection}

Dense semantic vectors for relationships, capturing the meaning of the connection between entities.

\begin{table}[H]
    \centering
    \caption{relationship\_embeddings Collection Properties}
    \begin{tabular}{|l|l|}
        \hline
        \textbf{Property} & \textbf{Value} \\
        \hline
        Total Documents & 22,737 \\
        \hline
        Embedding Model & all-MiniLM-L6-v2 \\
        \hline
        Vector Dimensions & 384 \\
        \hline
        Normalization & L2 (unit norm) \\
        \hline
        Memory per Vector & 1.5 KB (float32) \\
        \hline
        Total Memory & $\sim$ 33.6 MB \\
        \hline
    \end{tabular}
\end{table}

\section{Entity Relationship Diagram}

\begin{figure}[H]
\centering
\begin{tikzpicture}[node distance=2.5cm, auto, scale=0.85]
    % Define styles
    \tikzstyle{entity} = [draw, rectangle, fill=yellow!30, minimum width=2cm, minimum height=0.8cm]
    \tikzstyle{relation} = [draw, ellipse, fill=blue!30, minimum width=1.8cm]
    
    % Entities
    \node[entity] (tactic) {Tactic\\(TA0002)};
    \node[entity, below left=of tactic] (parent) {Technique\\(T1055)};
    \node[entity, below right=of tactic] (child) {Sub-Technique\\(T1055.004)};
    \node[entity, below=of parent] (group) {Threat Group\\(G0001)};
    \node[entity, right=of group] (adv) {CISA Advisory\\(aa24-060a)};
    
    % Relationships
    \node[relation, above=of parent, yshift=-0.5cm] (rt1) {belongs-to};
    \node[relation, above=of child, yshift=-0.5cm] (rt2) {sub-of};
    \node[relation, below=of parent, yshift=0.5cm] (rt3) {uses};
    \node[relation, below=of adv, yshift=0.5cm] (rt4) {uses\_in};
    
    % Arrows
    \draw[->] (parent) -- (rt1) -- (tactic);
    \draw[->] (child) -- (rt2) -- (parent);
    \draw[->] (group) -- (rt3) -- (parent);
    \draw[->] (parent) -- (rt4) -- (adv);
    
\end{tikzpicture}
\caption{Entity Relationship Diagram}
\end{figure}

\section{Entity Schema Examples}

\subsection{Attack Pattern Entity}

\begin{lstlisting}[language=json, caption=Example Attack Pattern Entity]
{
  "_key": "attack-pattern_890c9858-598c-401d-a4d5-c67ebcdd703a",
  "type": "attack-pattern",
  "external_id": "T1528",
  "name": "Steal Application Access Token",
  "description": "Adversaries may steal application access tokens...",
  "x_mitre_platforms": ["IaaS", "SaaS"],
  "x_mitre_data_sources": [
    "Application: Application access token creation",
    "Application: Application access token use"
  ],
  "version": "1.0",
  "created": "2020-01-01T00:00:00.000Z",
  "modified": "2023-08-01T00:00:00.000Z"
}
\end{lstlisting}

\subsection{CISA Advisory Entity}

\begin{lstlisting}[language=json, caption=Example CISA Advisory Entity]
{
  "_key": "cisa_advisory_aa24-060a",
  "type": "cisa_advisory",
  "external_id": "aa24-060a",
  "name": "CISA Alert: Phobos Ransomware Activity",
  "url": "https://www.cisa.gov/news-events/cybersecurity-advisories/aa24-060a",
  "cleantext": "Actions to take today to mitigate Phobos ransomware...",
  "rawtext": "Full advisory text here...",
  "created_date": "2024-02-29"
}
\end{lstlisting}

\subsection{Relationship Entity}

\begin{lstlisting}[language=json, caption=Example Relationship Entity]
{
  "_key": "rel_t1528_aa24060a",
  "_from": "entities/attack-pattern_890c9858-598c-401d-a4d5-c67ebcdd703a",
  "_to": "entities/cisa_advisory_aa24-060a",
  "relationship_type": "uses_in_cisa_advisory",
  "technique_id": "T1528",
  "advisory_id": "aa24-060a",
  "description": "T1528 technique is used in advisory aa24-060a"
}
\end{lstlisting}

\newpage

% ============================================================================
\chapter{Data Generation \& Triplet Extraction}
% ============================================================================

\section{Data Processing Pipeline Overview}

\begin{figure}[H]
\centering
\begin{tikzpicture}[node distance=2cm, auto, scale=0.85]
    % Input nodes
    \node[draw, rectangle, fill=cyan!30, minimum width=2.2cm, minimum height=0.8cm] (mitre) {MITRE JSON\\(14K STIX)};
    \node[draw, rectangle, fill=cyan!30, minimum width=2.2cm, minimum height=0.8cm, right=of mitre, xshift=0.8cm] (cisa) {CISA CSV\\(77 Advisories)};
    
    % Processing nodes
    \node[draw, circle, fill=lime!30, minimum width=1.5cm, below=of mitre] (parse1) {Parse};
    \node[draw, circle, fill=lime!30, minimum width=1.5cm, below=of cisa] (parse2) {Parse};
    \node[draw, rectangle, fill=orange!30, minimum width=2cm, minimum height=0.8cm, below=of parse1] (mitre_proc) {Normalize\\MITRE};
    \node[draw, rectangle, fill=orange!30, minimum width=2cm, minimum height=0.8cm, below=of parse2] (cisa_proc) {Extract\\CISA TTPs};
    \node[draw, rectangle, fill=yellow!30, minimum width=2cm, minimum height=0.8cm, below left=of mitre_proc] (gen_triplets) {Generate\\Triplets};
    
    % Output nodes
    \node[draw, rectangle, fill=magenta!20, minimum width=1.8cm, minimum height=0.8cm, below=of gen_triplets, yshift=-0.3cm] (ent) {Entities\\(24.5K)};
    \node[draw, rectangle, fill=magenta!20, minimum width=1.8cm, minimum height=0.8cm, right=of ent] (rel) {Relations\\(24.3K)};
    \node[draw, rectangle, fill=magenta!20, minimum width=1.8cm, minimum height=0.8cm, right=of rel] (emb) {Embeddings\\(47.3K)};
    
    % Database
    \node[draw, rectangle, fill=red!20, minimum width=3cm, minimum height=0.8cm, below=of rel, yshift=-0.5cm] (adb) {ArangoDB MITRE2kg};
    
    % Connections
    \draw[->] (mitre) -- (parse1);
    \draw[->] (cisa) -- (parse2);
    \draw[->] (parse1) -- (mitre_proc);
    \draw[->] (parse2) -- (cisa_proc);
    \draw[->] (mitre_proc) -- (gen_triplets);
    \draw[->] (cisa_proc) -- (gen_triplets);
    \draw[->] (gen_triplets) -- (ent);
    \draw[->] (gen_triplets) -- (rel);
    \draw[->] (gen_triplets) -- (emb);
    \draw[->] (ent) -- (adb);
    \draw[->] (rel) -- (adb);
    \draw[->] (emb) -- (adb);
    
\end{tikzpicture}
\caption{Complete Data Processing Pipeline}
\label{fig:data_pipeline}
\end{figure}

\section{Source 1: MITRE Enterprise Attack Framework}

\subsection{Input Data}

The \texttt{enterprise-attack.json} file (STIX 2.1 format) contains:

\begin{itemize}
    \item 14,000+ STIX objects
    \item Complete attack technique library
    \item Threat actor and malware relationships
    \item Mitigation and detection strategies
\end{itemize}

\subsection{Processing Steps}

\subsubsection{Step 1: Parse STIX Objects}

\begin{lstlisting}[caption=MITRE Data Parsing]
import json

with open('enterprise-attack.json', 'r') as f:
    mitre_data = json.load(f)

# Extract different object types
attack_patterns = [
    obj for obj in mitre_data['objects'] 
    if obj['type'] == 'attack-pattern'
]
tactics = [
    obj for obj in mitre_data['objects'] 
    if 'x-mitre-tactic' in obj['type']
]
threat_groups = [
    obj for obj in mitre_data['objects'] 
    if obj['type'] == 'intrusion-set'
]
relationships = [
    obj for obj in mitre_data['objects'] 
    if obj['type'] == 'relationship'
]
\end{lstlisting}

\subsubsection{Step 2: Extract Entity Objects}

The framework extracts different entity types:

\begin{table}[H]
    \centering
    \caption{MITRE Entity Extraction}
    \begin{tabular}{|l|l|l|}
        \hline
        \textbf{Object Type} & \textbf{Count} & \textbf{Example ID} \\
        \hline
        attack-pattern & 823 & T1055, T1055.001 \\
        \hline
        x-mitre-tactic & 14 & TA0002 \\
        \hline
        intrusion-set & 178 & G0001 \\
        \hline
        malware & 695 & S0005 \\
        \hline
        tool & 91 & S0002 \\
        \hline
    \end{tabular}
\end{table}

\subsubsection{Step 3: Normalize External IDs}

\begin{table}[H]
    \centering
    \caption{External ID Normalization}
    \begin{tabular}{|l|l|l|}
        \hline
        \textbf{Type} & \textbf{Parent Format} & \textbf{Subtechnique Format} \\
        \hline
        Attack Pattern & T1055 & T1055.001, T1055.002, ... \\
        \hline
        Tactic & TA0002 & (no subtechniques) \\
        \hline
        Group & G0001 & (no subtechniques) \\
        \hline
    \end{tabular}
\end{table}

\subsubsection{Step 4: Extract Relationships (Triplets)}

Relationships are extracted as subject-predicate-object triplets:

\begin{lstlisting}[caption=Triplet Extraction from MITRE]
# Example triplets
triplets = [
    # Subtechnique hierarchy
    (T1055.001, subtechnique-of, T1055),
    (T1055.002, subtechnique-of, T1055),
    
    # Tactic mapping
    (T1055, belongs-to-tactic, TA0002),
    (T1083, belongs-to-tactic, TA0007),
    
    # Group usage
    (G0001, uses, T1055),
    (G0001, uses, T1083),
    
    # Mitigations
    (T1055, mitigates, T1110),
    
    # Detections
    (T1055, detects, detection-strategy-001)
]
\end{lstlisting}

\subsection{Embedding Generation}

For each entity, a semantic embedding is generated:

\begin{lstlisting}[caption=Entity Embedding Generation, language=python]
from sentence_transformers import SentenceTransformer

model = SentenceTransformer('all-MiniLM-L6-v2')

# For attack pattern T1528
entity_text = (
    "Steal Application Access Token. "
    "Adversaries may steal application access tokens..."
)

# Generate 384-dimensional embedding
embedding = model.encode(entity_text, normalize_embeddings=True)

# embedding is now a numpy array of shape (384,)
# Each value is a float between -1 and 1
# L2 norm = 1.0 (normalized)
\end{lstlisting}

\section{Source 2: CISA Cybersecurity Advisories}

\subsection{CISA Integration Pipeline}

\begin{figure}[H]
\centering
\begin{tikzpicture}[node distance=2cm, auto, scale=0.8]
    % Input
    \node[draw, rectangle, fill=cyan!30, minimum width=2.5cm, minimum height=0.8cm] (cisa_csv) {CISA CSV\\(77 advisories)};
    
    % Processing steps
    \node[draw, rectangle, fill=lime!30, minimum width=2.5cm, minimum height=0.8cm, below=of cisa_csv] (parse_csv) {Parse CSV\\Rows};
    \node[draw, rectangle, fill=orange!30, minimum width=2.5cm, minimum height=0.8cm, below=of parse_csv] (extract_ttp) {Extract TTPs\\(T1234)};
    \node[draw, rectangle, fill=yellow!30, minimum width=2.5cm, minimum height=0.8cm, below=of extract_ttp] (match_ttp) {Match to Attack\\Patterns};
    \node[draw, rectangle, fill=magenta!20, minimum width=2.5cm, minimum height=0.8cm, below=of match_ttp] (create_adv) {Create Advisory\\Entities};
    \node[draw, rectangle, fill=pink!30, minimum width=2.5cm, minimum height=0.8cm, below=of create_adv] (link_rels) {Link Relationships};
    
    % Output
    \node[draw, rectangle, fill=red!20, minimum width=2.5cm, minimum height=0.8cm, below=of link_rels] (results) {74 Advisories\\1,605 Links};
    
    % Connections
    \draw[->] (cisa_csv) -- (parse_csv);
    \draw[->] (parse_csv) -- (extract_ttp);
    \draw[->] (extract_ttp) -- (match_ttp);
    \draw[->] (match_ttp) -- (create_adv);
    \draw[->] (create_adv) -- (link_rels);
    \draw[->] (link_rels) -- (results);
    
    % Stats annotation
    \node[right=of match_ttp, xshift=1.5cm] (stats) {\textbf{Match Rate}: 94.4\%};
    
\end{tikzpicture}
\caption{CISA Integration Pipeline}
\label{fig:cisa_pipeline}
\end{figure}

\subsection{Input Data}

The \texttt{CISA-crawl-rt-ttp-ct.csv} file contains:

\begin{table}[H]
    \centering
    \caption{CISA CSV Structure}
    \begin{tabular}{|p{2cm}|p{4cm}|p{7cm}|}
        \hline
        \textbf{Column} & \textbf{Example} & \textbf{Description} \\
        \hline
        RawText & Full advisory text... & Complete advisory content \\
        \hline
        CleanText & Cleaned advisory... & Preprocessed text \\
        \hline
        TTP & $\{$T1083, T1071.002$\}$ & Set of attack techniques \\
        \hline
        URL & https://www.cisa.gov/.../aa24-060a & Link to advisory \\
        \hline
    \end{tabular}
\end{table}

\begin{itemize}
    \item Total rows: 77 advisories
    \item Unique advisory IDs: 76
    \item Total TTP references: 1,701
    \item Unique T\# techniques: 432
\end{itemize}

\subsection{Processing Pipeline}

\subsubsection{Step 1: Load CSV Data}

\begin{lstlisting}[caption=CISA CSV Loading, language=python]
import csv

advisories = []
with open('CISA-crawl-rt-ttp-ct.csv', 'r') as f:
    reader = csv.DictReader(f)
    for row in reader:
        advisories.append({
            'url': row['URL'],
            'cleantext': row['CleanText'],
            'ttp_field': row['TTP']
        })

print(f"Loaded {len(advisories)} advisories")
\end{lstlisting}

\subsubsection{Step 2: Extract Advisory ID and TTPs}

\begin{lstlisting}[caption=Advisory ID and TTP Extraction, language=python]
import re
from collections import defaultdict

advisory_ttp_map = defaultdict(set)

for advisory in advisories:
    # Extract advisory ID from URL
    match = re.search(r'(aa\d{2}-\d+[a-z]?)', advisory['url'].lower())
    if match:
        advisory_id = match.group(1).lower()
        
        # Extract T# from TTP field
        ttps = set(re.findall(r'T\d+(?:\.\d+)?', advisory['ttp_field']))
        
        advisory_ttp_map[advisory_id].update(ttps)

print(f"Found {len(advisory_ttp_map)} unique advisories")
for adv_id, ttps in sorted(advisory_ttp_map.items())[:5]:
    print(f"  {adv_id}: {len(ttps)} techniques")
\end{lstlisting}

\subsubsection{Step 3: Create CISA Advisory Entities}

For each unique advisory ID, create an entity in ArangoDB:

\begin{lstlisting}[caption=CISA Entity Creation, language=python]
from arango import ArangoClient

db = ArangoClient(hosts='http://localhost:8529').db(
    'MITRE2kg', 
    username='root', 
    password='openSesame'
)

entities = db.collection('entities')

for advisory_id in advisory_ttp_map.keys():
    entity = {
        'type': 'cisa_advisory',
        'external_id': advisory_id,
        'name': f'CISA Advisory {advisory_id}',
        'url': f'https://www.cisa.gov/../{advisory_id}',
        'created_date': '2024-01-01'
    }
    
    entities.insert(entity)

print(f"Created {len(advisory_ttp_map)} CISA advisory entities")
\end{lstlisting}

\subsubsection{Step 4: Create Technique-to-Advisory Links}

For each TTP in each advisory, create a relationship:

\begin{lstlisting}[caption=Creating TTP-Advisory Links, language=python]
relationships = db.collection('relationships')
success_count = 0
failed_count = 0

for advisory_id, ttps in advisory_ttp_map.items():
    for ttp in ttps:
        # Find attack-pattern entity with this T#
        cursor = db.aql.execute(f'''
            FOR e IN entities
            FILTER e.type == "attack-pattern"
            FILTER e.external_id == "{ttp}"
            RETURN e._id
        ''')
        
        pattern_id = None
        for doc in cursor:
            pattern_id = doc
            break
        
        if pattern_id:
            # Find CISA advisory entity
            cursor = db.aql.execute(f'''
                FOR e IN entities
                FILTER e.type == "cisa_advisory"
                FILTER e.external_id == "{advisory_id}"
                RETURN e._id
            ''')
            
            advisory_entity_id = None
            for doc in cursor:
                advisory_entity_id = doc
                break
            
            if advisory_entity_id:
                # Create relationship
                rel = {
                    '_from': pattern_id,
                    '_to': advisory_entity_id,
                    'relationship_type': 'uses_in_cisa_advisory',
                    'technique_id': ttp,
                    'advisory_id': advisory_id
                }
                relationships.insert(rel)
                success_count += 1
        else:
            failed_count += 1

print(f"Created {success_count} relationships")
print(f"Failed: {failed_count}")
print(f"Success rate: {100*success_count/(success_count+failed_count):.1f}%")
\end{lstlisting}

\subsection{Result Statistics}

\begin{table}[H]
    \centering
    \caption{CISA Integration Results}
    \begin{tabular}{|l|r|}
        \hline
        \textbf{Metric} & \textbf{Value} \\
        \hline
        Advisories processed & 77 \\
        \hline
        Unique advisory IDs & 76 \\
        \hline
        CISA advisory entities created & 74 \\
        \hline
        Total TTP references in CSV & 1,701 \\
        \hline
        Unique T\# techniques referenced & 432 \\
        \hline
        T\# techniques matched in database & 405 (93.8\%) \\
        \hline
        T\# techniques not found & 27 (6.2\%) \\
        \hline
        Relationships created & 1,605 \\
        \hline
        Success rate & 94.4\% \\
        \hline
        Unique techniques linked & 368 \\
        \hline
    \end{tabular}
\end{table}

\newpage

% ============================================================================
\chapter{Semantic Embeddings}
% ============================================================================

\section{Embedding Model}

The system uses \texttt{all-MiniLM-L6-v2}, a lightweight sentence-transformer model:

\begin{itemize}
    \item \textbf{Input:} Text (unlimited length, but typically 512 tokens max)
    \item \textbf{Output:} 384-dimensional vector
    \item \textbf{Model Size:} $\sim$ 22 MB
    \item \textbf{Inference Speed:} $\sim$ 50-100 ms per text
    \item \textbf{Normalization:} L2 (Euclidean)
\end{itemize}

\section{Entity Embedding Generation}

\subsection{Process}

\begin{enumerate}
    \item Concatenate entity name + description
    \item Tokenize text (max 512 tokens)
    \item Pass through transformer model
    \item Generate 384-dimensional vector
    \item L2-normalize vector to unit norm
    \item Store in \texttt{entity\_embeddings} collection
\end{enumerate}

\subsection{Example: T1528 (Steal Application Access Token)}

\begin{lstlisting}[caption=T1528 Embedding Example, language=python]
entity = {
    "external_id": "T1528",
    "name": "Steal Application Access Token",
    "description": "Adversaries may steal application access tokens..."
}

text = f"{entity['name']}. {entity['description']}"
embedding = model.encode(text, normalize_embeddings=True)

# Result: 384-dimensional vector
first_10 = [-0.1404, 0.0750, 0.0211, -0.0856, -0.0047, 
            0.0105, 0.0409, 0.0686, 0.0476, -0.0241]

# Vector statistics
import numpy as np
arr = np.array(embedding)
print(f"Min: {arr.min():.4f}")  # -0.1404
print(f"Max: {arr.max():.4f}")  # 0.1452
print(f"Mean: {arr.mean():.4f}")  # 0.0005
print(f"Std: {arr.std():.4f}")  # 0.0510
print(f"L2 Norm: {np.linalg.norm(arr):.4f}")  # 1.0000
\end{lstlisting}

\section{Relationship Embedding Generation}

Relationship embeddings are generated by combining the embeddings of both connected entities:

\begin{lstlisting}[caption=Relationship Embedding, language=python]
# For relationship: T1055 --subtechnique-of--> T1055.001

# Get embeddings of both entities
emb_parent = model.encode("Process Injection")  # T1055
emb_child = model.encode("DLL Injection")  # T1055.001

# Concatenate embeddings and relationship type text
combined_text = (
    f"Process Injection subtechnique-of DLL Injection"
)

rel_embedding = model.encode(
    combined_text, 
    normalize_embeddings=True
)

# Result: 384-dimensional relationship embedding
\end{lstlisting}

\section{Embedding Statistics}

\begin{table}[H]
    \centering
    \caption{Embedding Collection Statistics}
    \begin{tabular}{|l|r|r|}
        \hline
        \textbf{Property} & \textbf{Entity Embeddings} & \textbf{Relationship Embeddings} \\
        \hline
        Total Vectors & 24,556 & 22,737 \\
        \hline
        Dimensions & 384 & 384 \\
        \hline
        Normalization & L2 & L2 \\
        \hline
        Min Value & $-0.1319$ & $-0.1289$ \\
        \hline
        Max Value & $0.1452$ & $0.1523$ \\
        \hline
        Memory per Vector & 1.5 KB & 1.5 KB \\
        \hline
        Total Memory & 37 MB & 33.6 MB \\
        \hline
        Grand Total & \multicolumn{2}{c|}{70.6 MB} \\
        \hline
    \end{tabular}
\end{table}

\newpage

% ============================================================================
\chapter{Query Examples \& Use Cases}
% ============================================================================

\section{Query Execution Flow}

\begin{figure}[H]
\centering
\begin{tikzpicture}[node distance=2.2cm, auto, scale=0.8]
    % User input
    \node[draw, rectangle, fill=cyan!30, minimum width=2.5cm, minimum height=0.8cm] (user) {User Query};
    
    % Processing steps
    \node[draw, rectangle, fill=lime!30, minimum width=2.5cm, minimum height=0.8cm, below=of user] (encode) {Encode to Vector\\(384-dim)};
    \node[draw, rectangle, fill=orange!30, minimum width=2.5cm, minimum height=0.8cm, below=of encode] (search) {Semantic Search\\(cosine similarity)};
    \node[draw, rectangle, fill=yellow!30, minimum width=2.5cm, minimum height=0.8cm, below=of search] (traverse) {Graph Traversal\\(relationships)};
    \node[draw, rectangle, fill=magenta!20, minimum width=2.5cm, minimum height=0.8cm, below=of traverse] (context) {Retrieve Context\\(from DB)};
    \node[draw, rectangle, fill=pink!30, minimum width=2.5cm, minimum height=0.8cm, below=of context] (llm) {Pass to LLM\\(Ollama)};
    
    % Output
    \node[draw, rectangle, fill=red!20, minimum width=2.5cm, minimum height=0.8cm, below=of llm] (response) {Generated Response};
    
    % Connections
    \draw[->] (user) -- (encode);
    \draw[->] (encode) -- (search);
    \draw[->] (search) -- (traverse);
    \draw[->] (traverse) -- (context);
    \draw[->] (context) -- (llm);
    \draw[->] (llm) -- (response);
    
\end{tikzpicture}
\caption{Query Execution Pipeline}
\label{fig:query_flow}
\end{figure}

\section{Semantic Search}

Find techniques related to a user query by computing similarity in embedding space.

\subsection{Example 1: Credential Theft}

\begin{lstlisting}[caption=Semantic Search Query, language=python]
from arango import ArangoClient
from sentence_transformers import SentenceTransformer

db = ArangoClient(hosts='http://localhost:8529').db('MITRE2kg', 
                                                     username='root', 
                                                     password='openSesame')
model = SentenceTransformer('all-MiniLM-L6-v2')

# Query: Find techniques related to stealing credentials
query = "steal authentication credentials"
query_embedding = model.encode(query, normalize_embeddings=True)

# Search in ArangoDB
results = db.aql.execute('''
    FOR embed IN entity_embeddings
    FILTER embed.embedding_dim == 384
    
    LET similarity = (
        REDUCE i IN 0..383 
        AGGREGATE dot = 0 
        INTO dot + (@query[i] * embed.embedding[i])
    )
    
    SORT similarity DESC
    LIMIT 10
    
    RETURN {
        entity_id: embed._key,
        similarity: similarity
    }
''', bind_vars={'query': query_embedding.tolist()})

# Top results:
# 1. T1528 (0.87) - Steal Application Access Token
# 2. T1187 (0.85) - Forced Authentication
# 3. T1110 (0.83) - Brute Force
# 4. T1056 (0.81) - Input Capture
# 5. T1021 (0.79) - Remote Service Session Initiation
\end{lstlisting}

\section{Graph Traversal}

Navigate the knowledge graph to find related entities.

\subsection{Example 2: Ransomware Techniques}

\begin{lstlisting}[caption=Find Techniques in Ransomware Advisories, language=aql]
// Find all attack techniques mentioned in ransomware-related advisories
FOR advisory IN entities
FILTER advisory.type == "cisa_advisory"
FILTER advisory.name LIKE "%ransomware%"

FOR rel IN relationships
FILTER rel._to == advisory._id
FILTER rel.relationship_type == "uses_in_cisa_advisory"

LET technique = DOCUMENT(rel._from)

RETURN {
    advisory: advisory.external_id,
    technique: technique.external_id,
    technique_name: technique.name
}
\end{lstlisting}

\section{Multi-hop Traversal}

Find indirect relationships through the graph.

\subsection{Example 3: APT Groups and Techniques}

\begin{lstlisting}[caption=Find APT Groups Using Phishing, language=aql]
// Find threat groups that use phishing techniques
FOR group IN entities
FILTER group.type == "intrusion-set"

FOR rel1 IN relationships
FILTER rel1._from == group._id
FILTER rel1.relationship_type == "uses"

LET technique = DOCUMENT(rel1._to)
FILTER technique.name LIKE "%phishing%"

RETURN {
    group: group.name,
    technique: technique.external_id,
    technique_name: technique.name
}
\end{lstlisting}

\section{CISA Advisory Lookup}

\subsection{Example 4: Techniques in Specific Advisory}

\begin{lstlisting}[caption=Techniques in Advisory aa24-060a, language=aql]
// Find all techniques mentioned in advisory aa24-060a (Phobos Ransomware)
FOR advisory IN entities
FILTER advisory.external_id == "aa24-060a"

FOR rel IN relationships
FILTER rel.relationship_type == "uses_in_cisa_advisory"
FILTER rel._to == advisory._id

LET technique = DOCUMENT(rel._from)

RETURN {
    technique_id: technique.external_id,
    technique_name: technique.name,
    advisory_id: advisory.external_id
}
\end{lstlisting}

\newpage

% ============================================================================
\chapter{Statistics \& Metrics}
% ============================================================================

\section{Overall Statistics}

\begin{table}[H]
    \centering
    \caption{Complete System Statistics}
    \begin{tabularx}{\textwidth}{|X|r|}
        \hline
        \textbf{Metric} & \textbf{Value} \\
        \hline
        Total Entities & 24,556 \\
        \hline
        Total Relationships & 24,342 \\
        \hline
        Attack Patterns (Parent) & 345 \\
        \hline
        Attack Patterns (Subtechniques) & 478 \\
        \hline
        Attack Patterns (Total) & 823 \\
        \hline
        CISA Advisories & 74 \\
        \hline
        Malware Families & 695 \\
        \hline
        Threat Groups (Intrusion Sets) & 178 \\
        \hline
        Tools & 91 \\
        \hline
        Campaigns & 52 \\
        \hline
        Tactics & 14 \\
        \hline
        Entity Embeddings & 24,556 \\
        \hline
        Relationship Embeddings & 22,737 \\
        \hline
        Total Embeddings & 47,293 \\
        \hline
        Embedding Dimensions & 384 \\
        \hline
        Total Memory (Embeddings) & $\sim$ 70.6 MB \\
        \hline
        Database Total Size & $\sim$ 100-120 MB \\
        \hline
    \end{tabular}
\end{table}

\section{Relationship Statistics}

\begin{figure}[H]
    \centering
    \caption{Relationship Type Distribution}
    \begin{tabular}{|l|r|r|}
        \hline
        \textbf{Relationship Type} & \textbf{Count} & \textbf{\% of Total} \\
        \hline
        uses & 17,270 & 70.9\% \\
        \hline
        referenced\_in\_cisa\_advisory & 1,619 & 6.6\% \\
        \hline
        uses\_in\_cisa\_advisory & 1,605 & 6.6\% \\
        \hline
        mitigates & 1,445 & 5.9\% \\
        \hline
        belongs-to-tactic & 1,071 & 4.4\% \\
        \hline
        detects & 691 & 2.8\% \\
        \hline
        subtechnique-of & 476 & 2.0\% \\
        \hline
        revoked-by & 140 & 0.6\% \\
        \hline
        attributed-to & 25 & 0.1\% \\
        \hline
        \textbf{TOTAL} & \textbf{24,342} & \textbf{100\%} \\
        \hline
    \end{tabular}
\end{figure}

\section{CISA Integration Metrics}

\begin{table}[H]
    \centering
    \caption{CISA Integration Statistics}
    \begin{tabular}{|l|r|}
        \hline
        \textbf{Metric} & \textbf{Value} \\
        \hline
        Advisories in CSV & 77 \\
        \hline
        Unique Advisory IDs & 76 \\
        \hline
        Advisory Entities Created & 74 \\
        \hline
        Coverage Rate & 98.7\% \\
        \hline
        Total TTP References & 1,701 \\
        \hline
        Unique T\# Techniques & 432 \\
        \hline
        Techniques Matched in DB & 405 \\
        \hline
        Match Rate & 93.8\% \\
        \hline
        TTP-to-Advisory Links & 1,605 \\
        \hline
        Success Rate & 94.4\% \\
        \hline
        Unique Techniques Linked & 368 \\
        \hline
        Avg Techniques per Advisory & 22.2 \\
        \hline
        Max Techniques per Advisory & 146 \\
        \hline
        Min Techniques per Advisory & 0 \\
        \hline
    \end{tabular}
\end{table}

\section{Graph Density Metrics}

\begin{table}[H]
    \centering
    \caption{Graph Connectivity}
    \begin{tabular}{|l|r|}
        \hline
        \textbf{Metric} & \textbf{Value} \\
        \hline
        Average Connections per Entity & 0.99 \\
        \hline
        Maximum Connections (Entity) & 17,270 \\
        \hline
        Minimum Connections & 0 \\
        \hline
        Graph Density & 0.0000405 \\
        \hline
        Average Path Length & 3.5 \\
        \hline
    \end{tabular}
\end{table}

\newpage

% ============================================================================
\chapter{Technical Architecture}
% ============================================================================

\section{System Architecture}

\begin{table}[H]
    \centering
    \caption{Technology Stack}
    \begin{tabular}{|l|l|l|}
        \hline
        \textbf{Layer} & \textbf{Component} & \textbf{Technology} \\
        \hline
        \multirow{2}{*}{Data Source} & MITRE Framework & enterprise-attack.json \\
         & CISA Advisories & CISA-crawl-rt-ttp-ct.csv \\
        \hline
        \multirow{2}{*}{Storage} & Graph Database & ArangoDB 3.x \\
         & Collections & entities, relationships, embeddings \\
        \hline
        \multirow{3}{*}{Processing} & Data Extraction & Python 3.12 \\
         & Embeddings & sentence-transformers \\
         & Database Client & python-arango \\
        \hline
        \multirow{2}{*}{Application} & Query Engine & Custom RAG Engine \\
         & LLM & Ollama (llama3.1:8b) \\
        \hline
    \end{tabular}
\end{table}

\section{ArangoDB Configuration}

\begin{table}[H]
    \centering
    \caption{ArangoDB Setup}
    \begin{tabular}{|l|l|}
        \hline
        \textbf{Parameter} & \textbf{Value} \\
        \hline
        Host & http://localhost:8529 \\
        \hline
        Database & MITRE2kg \\
        \hline
        Username & root \\
        \hline
        Password & openSesame \\
        \hline
        Database Model & Multi-model (Document + Graph) \\
        \hline
        Query Language & ArangoDB Query Language (AQL) \\
        \hline
    \end{tabular}
\end{table}

\section{Python Libraries}

\begin{lstlisting}[caption=Required Python Packages, language=bash]
# Core dependencies
python-arango>=7.0        # ArangoDB client
sentence-transformers     # Embedding model
numpy>=1.20              # Numerical computing
pandas>=1.3              # Data manipulation
requests>=2.25           # HTTP requests (for Ollama)

# Optional dependencies
torch>=1.9               # PyTorch (for transformers)
scikit-learn>=0.24       # Machine learning utilities
\end{lstlisting}

\section{Ollama LLM Integration}

\begin{table}[H]
    \centering
    \caption{Ollama Configuration}
    \begin{tabular}{|l|l|}
        \hline
        \textbf{Parameter} & \textbf{Value} \\
        \hline
        Endpoint & http://localhost:11434 \\
        \hline
        Model & llama3.1:8b \\
        \hline
        Model Size & $\sim$ 5-8 GB \\
        \hline
        Inference Speed & 2-5 seconds per query \\
        \hline
        Context Window & 8,192 tokens \\
        \hline
        Memory Required & 8-16 GB RAM \\
        \hline
    \end{tabular}
\end{table}

\section{Performance Characteristics}

\begin{table}[H]
    \centering
    \caption{Query Performance Metrics}
    \begin{tabular}{|l|r|}
        \hline
        \textbf{Operation} & \textbf{Latency} \\
        \hline
        Entity lookup (by ID) & < 10 ms \\
        \hline
        Semantic search (1000 entities) & 100-200 ms \\
        \hline
        Graph traversal (2-hop) & 50-100 ms \\
        \hline
        Graph traversal (3-4 hops) & 200-500 ms \\
        \hline
        Query encoding & 50-100 ms \\
        \hline
        Context retrieval & 100-200 ms \\
        \hline
        LLM inference & 2-5 seconds \\
        \hline
        Total end-to-end & 3-7 seconds \\
        \hline
    \end{tabular}
\end{table}

\section{Memory Requirements}

\begin{table}[H]
    \centering
    \caption{System Memory Usage}
    \begin{tabular}{|l|r|}
        \hline
        \textbf{Component} & \textbf{Memory} \\
        \hline
        ArangoDB Process & 300-500 MB \\
        \hline
        Python RAG Engine & 200-300 MB \\
        \hline
        Ollama (llama3.1:8b) & 5-8 GB \\
        \hline
        OS and Other & 2-3 GB \\
        \hline
        \textbf{Total System} & \textbf{6-9 GB RAM} \\
        \hline
    \end{tabular}
\end{table}

\newpage

% ============================================================================
\chapter{Implementation Details}
% ============================================================================

\section{Data Flow Diagram}

The complete data processing pipeline:

\begin{enumerate}
    \item \textbf{Data Ingestion:} Load enterprise-attack.json and CISA-crawl CSV
    \item \textbf{Entity Extraction:} Parse STIX objects and CSV rows
    \item \textbf{Entity Normalization:} Standardize IDs and formats
    \item \textbf{Relationship Extraction:} Generate triplets from relationships
    \item \textbf{Database Insertion:} Store entities and relationships in ArangoDB
    \item \textbf{Embedding Generation:} Create 384-dimensional vectors for all entities
    \item \textbf{Embedding Storage:} Store vectors in embedding collections
    \item \textbf{Verification:} Cross-validate data integrity and completeness
    \item \textbf{Query Interface:} Enable RAG queries and semantic search
\end{enumerate}

\section{Key Algorithms}

\subsection{TTP Extraction Algorithm}

\begin{lstlisting}[caption=TTP Extraction from CISA Advisory, language=python]
import re

def extract_ttp_ids(ttp_field):
    """Extract T# technique IDs from TTP field."""
    return set(re.findall(r'T\d+(?:\.\d+)?', str(ttp_field)))

def extract_advisory_id(url):
    """Extract advisory ID from URL."""
    match = re.search(r'(aa\d{2}-\d+[a-z]?)', url.lower())
    return match.group(1).lower() if match else None

# Usage
advisory_url = "https://www.cisa.gov/.../aa24-060a"
ttp_field = "{T1083, T1071.002, T1055.004, ...}"

advisory_id = extract_advisory_id(advisory_url)  # "aa24-060a"
ttps = extract_ttp_ids(ttp_field)  # {"T1083", "T1071.002", ...}
\end{lstlisting}

\subsection{Semantic Similarity Search}

\begin{lstlisting}[caption=Cosine Similarity Computation, language=python]
import numpy as np

def cosine_similarity(vec1, vec2):
    """Compute cosine similarity between two vectors."""
    # Both vectors are L2-normalized, so:
    # similarity = dot product
    return np.dot(vec1, vec2)

# Example
query_embedding = model.encode("steal credentials")  # 384-dim
entity_embedding = model.encode("Steal Application Access Token")  # 384-dim

similarity = cosine_similarity(query_embedding, entity_embedding)
# Result: 0.87 (high similarity)
\end{lstlisting}

\subsection{Graph Traversal}

\begin{lstlisting}[caption=Multi-hop Graph Traversal, language=python]
def graph_traversal(start_entity_id, relationship_types, max_depth=3):
    """Traverse graph from starting entity."""
    visited = set()
    queue = [(start_entity_id, 0)]
    results = []
    
    while queue:
        entity_id, depth = queue.pop(0)
        
        if entity_id in visited or depth > max_depth:
            continue
        
        visited.add(entity_id)
        
        # Query related entities
        cursor = db.aql.execute(f'''
            FOR rel IN relationships
            FILTER rel._from == "{entity_id}"
            FILTER rel.relationship_type IN {relationship_types}
            RETURN {{"to_id": rel._to, "rel_type": rel.relationship_type}}
        ''')
        
        for doc in cursor:
            queue.append((doc['to_id'], depth + 1))
            results.append(doc)
    
    return results
\end{lstlisting}

\newpage

% ============================================================================
\chapter{Conclusions \& Future Work}
% ============================================================================

\section{Key Achievements}

\begin{enumerate}
    \item \textbf{Unified Knowledge Graph:} Successfully integrated MITRE ATT\&CK framework 
    with real-world CISA advisories, creating a comprehensive cybersecurity knowledge base.
    
    \item \textbf{Semantic Embeddings:} Generated 47,293 semantic vectors enabling similarity 
    search and contextual understanding of attack techniques.
    
    \item \textbf{CISA Integration:} Created 1,605 links between 368 attack techniques and 
    74 CISA advisories with 94.4\% success rate.
    
    \item \textbf{Graph Database Implementation:} Implemented multi-model graph database with 
    24,556 entities and 24,342 relationships for efficient querying.
    
    \item \textbf{LLM Integration:} Enabled AI-powered question answering using Ollama 
    with graph context injection.
    
    \item \textbf{Scalable Architecture:} Designed system supporting 47K+ embeddings and 
    50K+ entities with sub-second query latency.
\end{enumerate}

\section{Use Cases Enabled}

\begin{description}
    \item[Incident Response:] Quick lookup of techniques and advisories related to detected attacks
    \item[Threat Intelligence:] Identify threat actors and their attack patterns
    \item[Vulnerability Assessment:] Link vulnerabilities to attack techniques and mitigations
    \item[Security Automation:] Query techniques, mitigations, and detections programmatically
    \item[Research \& Analysis:] Explore relationships in cybersecurity threat landscape
\end{description}

\section{Future Enhancements}

\begin{itemize}
    \item Integration of additional threat intelligence sources (CVE, exploits, vulnerabilities)
    \item Advanced graph algorithms (centrality analysis, community detection)
    \item Real-time threat intelligence updates
    \item Custom fine-tuning of embedding models for cybersecurity domain
    \item Web API and visualization dashboard
    \item Integration with security tools (SIEM, IDS, endpoint protection)
\end{itemize}

\section{Technical Debt \& Limitations}

\begin{itemize}
    \item Some CISA advisory TTPs not matched (6.2\% mismatch rate) due to format variations
    \item Embedding model generic (not domain-specialized for cybersecurity)
    \item Limited real-time update mechanisms
    \item No access control or role-based permissions
    \item Single-instance deployment (not clustered)
\end{itemize}

\newpage

% ============================================================================
% Appendices
% ============================================================================

\appendix

\chapter{File Structure}

\begin{lstlisting}[caption=Project File Structure, language=bash]
/home/vasanthiyer-gpu/
├── enterprise-attack.json              # MITRE framework data
├── CISA-crawl-rt-ttp-ct.csv           # CISA advisories
├── arangodb_rag_query_engine.py        # RAG query implementation
├── ollama_rag_integration.py           # Ollama LLM integration
├── fast_embeddings.py                  # Embedding generation
├── create_ttp_links.py                 # TTP-advisory linking
├── query_ttp.py                        # Query tools
├── MITRE2KG_GRAPHDB_DOCUMENTATION.tex # This document
├── GRAPHDB_SCHEMA_SUMMARY.md           # Schema documentation
├── GRAPHDB_ARCHITECTURE.md             # Architecture diagrams
└── GRAPHDB_IMPLEMENTATION_DETAILS.md   # Implementation guide
\end{lstlisting}

\chapter{Database Queries}

\section{Count Entities by Type}

\begin{lstlisting}[language=aql, caption=AQL Query: Entity Counts]
FOR e IN entities
COLLECT type = e.type WITH COUNT INTO count
RETURN {type: type, count: count}
ORDER BY count DESC
\end{lstlisting}

\section{Find All Techniques in Advisory}

\begin{lstlisting}[language=aql, caption=AQL Query: Advisory Techniques]
FOR advisory IN entities
FILTER advisory.external_id == "aa24-060a"
FOR rel IN relationships
FILTER rel.relationship_type == "uses_in_cisa_advisory"
FILTER rel._to == advisory._id
LET technique = DOCUMENT(rel._from)
RETURN {
  technique_id: technique.external_id,
  technique_name: technique.name
}
\end{lstlisting}

\section{Semantic Similarity Search}

\begin{lstlisting}[language=python, caption=Python: Semantic Search]
from sentence_transformers import SentenceTransformer
from arango import ArangoClient

model = SentenceTransformer('all-MiniLM-L6-v2')
db = ArangoClient(hosts='http://localhost:8529').db('MITRE2kg')

# Query
query_embedding = model.encode("steal credentials")

# Search
cursor = db.aql.execute('''
FOR embed IN entity_embeddings
FILTER embed._id LIKE "entity_embeddings/attack-pattern%"
LET similarity = REDUCE i IN 0..383 
    AGGREGATE dot = 0 
    INTO dot + (@query[i] * embed.embedding[i])
SORT similarity DESC
LIMIT 10
RETURN {entity: embed._key, similarity: similarity}
''', bind_vars={'query': query_embedding.tolist()})
\end{lstlisting}

\chapter{Additional Resources}

\begin{itemize}
    \item MITRE ATT\&CK Framework: \url{https://attack.mitre.org}
    \item CISA Advisories: \url{https://www.cisa.gov/cybersecurity-advisories}
    \item ArangoDB Documentation: \url{https://www.arangodb.com/docs}
    \item Sentence Transformers: \url{https://www.sbert.net}
    \item Ollama: \url{https://ollama.ai}
\end{itemize}

\end{document}
